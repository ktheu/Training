\documentclass[addpoints]{exam}
\usepackage[utf8]{inputenc}
\usepackage[ngerman]{babel}
\usepackage{listings} 
\usepackage{babel}
\usepackage[top=1.5cm,bottom=0.5cm,headsep=0.5cm,headheight=3cm,%   
left=1.5cm,right=1.5cm]{geometry}                      
                         
\usepackage[T1]{fontenc}                               
\usepackage{booktabs} % schöne Tabellen                
\usepackage{graphicx}                                  
\usepackage{csquotes} % Anführungszeichen              
\usepackage{paralist} % kompakte Aufzählungen          
\usepackage{amsmath,textcomp,tikz} %diverses          
\usepackage{eso-pic} % Bilder im Hintergrund          
\usepackage{mdframed} % Boxen         
\usepackage{multirow}    

    
\newmdenv[linecolor=black,backgroundcolor=gray!15,    
frametitle={Punktverteilung},leftmargin=1cm,
rightmargin=1cm]{infobox}

\lstset{language=Python, tabsize=4, basicstyle=\footnotesize, showstringspaces=false, mathescape=true}  
\lstset{literate=%
  {Ö}{{\"O}}1
  {Ä}{{\"A}}1
  {Ü}{{\"U}}1
  {ß}{{\ss}}1
  {ü}{{\"u}}1
  {ä}{{\"a}}1
  {ö}{{\"o}}1
}
\begin{document}
\pointpoints{Punkt}{Punkte}
\bonuspointpoints{Bonuspunkt}{Bonuspunkte}                     
\renewcommand{\solutiontitle}{\noindent\textbf{Lösung:}%       
\enspace}                                                      
                                                               
\chqword{Frage}                                                
\chpgword{Seite}                                               
\chpword{Punkte}                                               
\chbpword{Bonus Punkte}                                        
\chsword{Erreicht}                                            
\chtword{Gesamt}                                              
\hpword{Punkte:} % Punktetabelle                              
\hsword{Ergebnis:}                                            
\hqword{Aufgabe:}                                             
\htword{Summe:}      
\cellwidth{1.5em}                                         
%\begin{center}
%\fbox{\fbox{\parbox{5.5in}{\centering
%Informatik-Klausur}}}
%\end{center}
%
%\vspace{5mm}
%
%\makebox[\textwidth]{Name:\enspace\hrulefill}
\pagestyle{headandfoot}
\runningheadrule

\newcommand\Vtextvisiblespace[1][.3em]{%
  \mbox{\kern.06em\vrule height.3ex}%
  \vbox{\hrule width#1}%
  \hbox{\vrule height.3ex}}

\firstpageheader{Lösungen Biberaufgaben - Blatt 1}{Name: }{ }

%\printanswers
\begin{questions}

\question
Bergsteiger
\begin{solutionbox}{4cm}
5,6,7
\end{solutionbox}

\question
Bienenflug
\begin{solutionbox}{4cm}
1,3,5,8
\end{solutionbox}

\question
Bequeme Biber
\begin{solutionbox}{4cm}
6-8
\end{solutionbox}

\question
Blumenkasten
\begin{solutionbox}{4cm}
\begin{lstlisting} 
R G R
G R G
R G O
\end{lstlisting} 
29 Punkte
\end{solutionbox}

\question
In einem Zug
\begin{solutionbox}{4cm}
C
\end{solutionbox}






% -------------------------------------------------
\end{questions}
\end{document}